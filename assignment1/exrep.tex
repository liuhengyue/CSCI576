\documentclass[10pt]{report}            % Report class in 11 points
\usepackage{mathtools}
\usepackage{amssymb,amsmath}
\parindent0pt  \parskip10pt             % make block paragraphs
\raggedright                            % do not right-justify
\title{\bf CSCI576 Assignment 1 Report}  % Supply information

\author{Hengyue Liu\\USC ID: 4107-2966-75}              %   for the title page.
\date{\today}                           %   Use current date.

\begin{document}                        % End of preamble, start of text.
\maketitle                              % Print title page.
\pagenumbering{roman}                   % roman page number for toc
\setcounter{page}{2}                    % make it start with "ii"
\tableofcontents                        % Print table of contents
\renewcommand{\chaptername}{Part}
%\renewcommand{\thechapter}{\Alph{chapter}}
\chapter{Written Questions}                % Print a "chapter" heading
\pagenumbering{arabic}                  % Start text with arabic 1
Most of this example applies to \texttt{article} and \texttt{book} classes
as well as to \texttt{report} class. In \texttt{article} class, however,
the default position for the title information is at the top of
the first text page rather than on a separate page. Also, it is
not usual to request a table of contents with \texttt{article} class.
 
\section*{Q 1}                  % Print a "section" heading
Solutions:
\begin{itemize}
\item                           % The following text will be \begin{quote}
 $N_{l}=450$\\                                %    set off and indented.
 $N_{p}=520$\\
 $N_{FPS}=25$\\ 
 P=12 bits per pixel for 4:2:0 scheme\\
 $N_l \cdot N_p \cdot N_{FPS} \cdot P = 7.02\times10^7$ bits/s.\\
 So, the bit-rate produced by the camera is $7.02\times10^7$ bits/s.\\
\item

                             % End of indented text
 \end{itemize}   
But note that---unlike the \texttt{book} and \texttt{report} classes---the
\texttt{article} class does not have a ``chapter" command.
 
\end{document}                          % The required last line
